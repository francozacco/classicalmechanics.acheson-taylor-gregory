\documentclass[11pt]{article}
\usepackage{amssymb}
\usepackage{amsthm}
\usepackage{enumitem}
\usepackage{amsmath}
\usepackage{bm}
\usepackage{adjustbox}
\usepackage{mathrsfs}

\title{\textbf{Solved selected problems of From Calculus to Chaos - Acheson}}
\author{Franco Zacco}
\date{}

\addtolength{\topmargin}{-3cm}
\addtolength{\textheight}{3cm}

\begin{document}


\maketitle
\thispagestyle{empty}

\section*{Chapter 3 - Ordinary differential equations}

	\begin{proof}{\textbf{3.2}}
		The equation presented can be classified as a nonlinear non-autonomous
		differential equation, and the equation is of the form
		$$\frac{dx}{dt}=g(x)h(t)$$
		where $g(x)=x^{2}$ and $h(t)=\frac{1}{1+t}$ which can be rewritten to
		the form
		$$\frac{1}{g(x)}\frac{dx}{dt}=h(t)$$
		as
		$$\frac{1}{x^{2}}\frac{dx}{dt} = \frac{1}{1+t}$$
		then integrating both sides with respect to $t$
		$$\int \frac{1}{x^2}\frac{dx}{dt}dt = \int \frac{1}{1+t}dt$$
		$$\int \frac{1}{x^2}dx = \int \frac{1}{1+t}dt$$
		$$-\frac{1}{x} = log(1+t) +C$$
		$$x = \frac{1}{C - log(1+t)}$$
		Subject to $x = 1$ when $t = 0$ then $C=1$ so we have
		$$x = \frac{1}{1 - log(1+t)}$$
		therefore the "blow up" of the solution happens when $t$ approaches
		$e-1$.
	\end{proof}
	\begin{proof}{\textbf{3.4}}
		
	\end{proof}
\end{document}























