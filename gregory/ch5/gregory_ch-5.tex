\documentclass[11pt]{article}
\usepackage{amssymb}
\usepackage{amsthm}
\usepackage{enumitem}
\usepackage{amsmath}
\usepackage{bm}
\usepackage{adjustbox}
\usepackage{mathrsfs}
\usepackage{graphicx}
\usepackage{siunitx}
\usepackage[mathscr]{euscript}

\title{\textbf{Solved selected problems of Classical Mechanics - Gregory}}
\author{Franco Zacco}
\date{}

\addtolength{\topmargin}{-3cm}
\addtolength{\textheight}{3cm}

\newcommand{\hatr}{\bm{\hat{r}}}
\newcommand{\hatx}{\bm{\hat{x}}}
\newcommand{\haty}{\bm{\hat{y}}}
\newcommand{\hatz}{\bm{\hat{z}}}
\newcommand{\hatth}{\bm{\hat{\theta}}}
\newcommand{\hatphi}{\bm{\hat{\phi}}}
\newcommand{\hatrho}{\bm{\hat{\rho}}}
\theoremstyle{definition}
\newtheorem*{solution*}{Solution}

\begin{document}
\maketitle
\thispagestyle{empty}

\section*{Chapter 5 - Linear Oscillations and normal modes}

	\begin{proof}{\textbf{5.2}}
    \begin{itemize}
        \item [(i)] When the two springs are in parallel the restoring force
        of each one is summed and they have a common strain given by $x$, so
        the equivalent spring constant or strength in this situation is given
        by
        \begin{align*}
            \alpha_p &= \frac{T_1 + T_2}{x} \\
                &= \frac{T_1}{x} + \frac{T_2}{x} \\
                &= \alpha_1 + \alpha_2 \\
                &= m\Omega_1^2 + m\Omega_2^2
        \end{align*}
        therefore the angular frecuency when the body is suspended with the
        springs in parallel is given by
        \begin{align*}
           m\Omega_p^2 &= m\Omega_1^2 + m\Omega_2^2 \\
           \Omega_p &= \sqrt{\Omega_1^2 + \Omega_2^2}
        \end{align*}
        \item[(ii)] In the case that the two springs are in series, one
        attached to the other the restoring force is common but the strain
        is going to be different for the two springs meaning that the the total
        strain is going to be $x_1 + x_2$, then
        \begin{align*}
            \alpha_s &= \frac{T}{x_1 + x_2} \\
                &= \frac{1}{\frac{x_1 + x_2}{T}} \\
                &= \frac{1}{\frac{x_1}{T} + \frac{x_2}{T}} \\
                &= \frac{1}{\frac{1}{\alpha_1} + \frac{1}{\alpha_2}} \\
                &= \frac{\alpha_1\alpha_2}{\alpha_1 + \alpha_2}
        \end{align*}
        therefore the angular frecuency when the body is suspended with the
        springs attached in series is given by
        \begin{align*}                                
            m\Omega_s^2 &= \frac{m^2\Omega_1^2\Omega_2^2}{m(\Omega_1^2 + \Omega_2^2)} \\
            \Omega_s &= \frac{\Omega_1\Omega_2}{\sqrt{\Omega_1^2 + \Omega_2^2}}
        \end{align*}
        \item [(iii)] From the above equations we can write that
        \begin{align*}
            \frac{\Omega_p}{\Omega_s} &= \frac{\Omega_1^2 + \Omega_2^2}{\Omega_1\Omega_2} \\
                &= \frac{\Omega_1^2 + \Omega_2^2 + 2\Omega_1\Omega_2 -2\Omega_1\Omega_2}{\Omega_1\Omega_2} \\
                &= \frac{(\Omega_1 - \Omega_2)^2 + 2\Omega_1\Omega_2}{\Omega_1\Omega_2} \\
                &= \frac{(\Omega_1 - \Omega_2)^2}{\Omega_1\Omega_2} + 2 \\
                &\geq 2
        \end{align*}
        Therefore, since $\frac{(\Omega_1 - \Omega_2)^2}{\Omega_1\Omega_2}$ is
        positive $\Omega_p$ is at least twice $\Omega_s$. 
    \end{itemize}
    \end{proof}
	\begin{proof}{\textbf{5.5}}
    \begin{itemize}
        \item [(a)] We know that the general real solution for an oscillator is
        given by
        $$x = e^{-Kt}(A\cos{\Omega_D t} + B\sin{\Omega_D t})$$
        where $A$ and $B$ are constants.
        
        We know that when $t=0$ then $x=a$ so replacing values we have the
        value for $A$ as
        \begin{align*}
            a &= e^0(A\cos 0 + B \sin 0) \\
            a &= A
        \end{align*}
        Derivating $x$ with respect to $t$ we have that
        $$\dot{x} = -Ke^{-Kt}(A\cos{\Omega_D t} + B\sin{\Omega_D t}) +
            e^{-Kt}(-A\Omega_D\sin{\Omega_D t} + B\Omega_D\cos{\Omega_D t})$$
        and we know that the particle starts from rest so when $t=0$ then
        $\dot{x} = 0$, so replacing values we have
        \begin{align*}
            0 &= -Ke^0(A\cos0 + B\sin0) + e^0(-A\Omega_D\sin0 + \Omega_DB\cos0) \\
            0 &= -KA + \Omega_D B \\
            B &= \frac{aK}{\Omega_D} 
        \end{align*}
        Finally, replacing the values for $A$ and $B$, we have that
        $$x = ae^{-Kt}\left(\cos{\Omega_D t} + \frac{K}{\Omega_D}\sin{\Omega_D t}\right)$$
        which is what we wanted.
        \item [(b)] To find the turning points of $x(t)$ we need to derivate
        $x$ and make it equal to $0$, since when $\dot{x} = 0$ we are at a
        turning point, then
        \begin{align*}
            \dot{x} &= -aKe^{-Kt}(\cos{\Omega_D t} + \frac{K}{\Omega_D}\sin{\Omega_D t})
                + a e^{-Kt}(-\Omega_D\sin{\Omega_D t} + \frac{K}{\Omega_D}\Omega_D\cos{\Omega_D t}) \\
                &= ae^{-Kt}(-K\cos{\Omega_D t} -\frac{K}{\Omega_D}\sin{\Omega_D t}
                -\Omega_D\sin{\Omega_D t} + K\cos{\Omega_D t}) \\
                &= ae^{-Kt}(-\frac{K}{\Omega_D}-\Omega_D)\sin\Omega_Dt
        \end{align*}
        So when $\sin\Omega_D t = 0$ we are at a turning point of $x$ which
        means that $t$ needs to be
        $$t = \frac{k\pi}{\Omega_D}$$
        where $k$ is an integer and $k \geq 0$.
        \item [(c)] According to the previous result the maximum value of $x$
        should happen at $t = 0, 2\pi/\Omega_D, 4\pi/\Omega_D, ...$ then the
        values of $x$ are
        $$x = a,~ae^{-\frac{2\pi K}{\Omega_D}},~ae^{-\frac{4\pi K}{\Omega_D}},~...$$
        therefore, the ratio between maximums is $e^{-\frac{2\pi K}{\Omega_D}}$
        \item [(d)] We know that the period of the oscillations in this case is
        defined as
        $$\tau = \frac{2\pi}{\Omega_D}$$
        then
        $$\Omega_D = \frac{2}{5}\pi$$
        on the other hand, the ratio between maximums is 3:1 so
        $$e^{-\frac{2\pi K}{\Omega_D}} = \frac{1}{3}$$
        then
        \begin{align*}
            e^{-\frac{2\pi K}{\frac{2\pi}{5}}} &= \frac{1}{3} \\
            e^{-5K} &= \frac{1}{3} \\
            -5K &= \log{\frac{1}{3}} \\
            K &= \frac{\log{3}}{5} \\
            \beta &= 4\log{3} = 4.39~\si{Nsm^{-1}} 
        \end{align*}
        finally we calculate $\alpha$ as
        \begin{align*}
            \Omega_D &= \frac{2}{5}\pi \\
            \Omega^2-K^2 &= \frac{4}{25}\pi^2 \\
            \Omega^2 &= \frac{4}{25}\pi^2 +\frac{(\log{3)^2}}{25} \\
            \alpha &= 10\left(\frac{4\pi^2 +(\log{3})^2}{25}\right) = 16.27~\si{Nm^{-1}} 
        \end{align*}
    \end{itemize}
    \end{proof}
	\begin{proof}{\textbf{5.6}}
        In the case of critical damping we want to find the solution to the
        SHM equation
        $$\frac{d^2x}{dt^2} + 2K\frac{dx}{dt} + \Omega^2x = 0$$
        when $K = \Omega$, so we proceed by seeking for solutions of the form
        $x = e^{-\lambda t}$. Then $\lambda$ must satisfy the equation
        $$\lambda^2 + 2K\lambda + \Omega^2 = 0$$
        that is
        \begin{align*}
            (\lambda + K)^2 &= K^2 - \Omega^2 \\
                            &= 0
        \end{align*}
        then $\lambda = -K = -\Omega$, so the set of solutions is given by
        \begin{equation*}
            x = 
            \begin{cases}
                e^{-K t} \\
                te^{-K t}
            \end{cases}
        \end{equation*}
        The general solution of the crical damped SHM equation in this case is
        therefore
        $$x = e^{-Kt}(A + Bt)$$
        when applying the initial conditions that $x=a$ and $\dot{x} =0 $ when
        $t=0$ we have that $A = a$ from $x(0) = a$ and $B = Ka = \Omega a$ from
        $\dot{x}(0) = 0$ therefore
        $$x = ae^{-\Omega t}(1 + \Omega t)$$
    \end{proof}
\cleardoublepage
	\begin{proof}{\textbf{5.7}}
        From problem 5.5 we know that the sub-critical solution for the SHM
        equation with the initial conditions $x=a$ when $t=0$ is
        $$x = ae^{-Kt}\left(\cos{\Omega_D t} + \frac{K}{\Omega_D}\sin{\Omega_D t}\right)$$
        also we know that the first minimum of this equation happens at 
        $$t = \frac{\pi}{\Omega_D} \quad\quad\text{then}\quad\quad 
            x(\pi/\Omega_D) = -ae^{-\frac{K\pi}{\Omega_D}}$$
        and we want that the $x = -\epsilon a$ then
        \begin{align*}
            -\epsilon a &= -ae^{-\frac{K\pi}{\Omega_D}} \\
               \epsilon &= e^{-\frac{K\pi}{\Omega_D}} \\
            \log{1/\epsilon} &= \frac{K\pi}{\sqrt{\Omega^2 - K^2}} \\
            \log^2{1/\epsilon} &= \frac{K^2\pi^2}{K^2(\frac{\Omega^2}{K^2} - 1)} \\
            \log^2{1/\epsilon} &= \frac{\pi^2}{\frac{\Omega^2}{K^2} - 1} \\
            \frac{\Omega^2}{K^2} - 1 &= \frac{\pi^2}{\log^2{1/\epsilon}} \\
            K &= \frac{\Omega}{\sqrt{\frac{\pi^2}{\log^2{1/\epsilon}} + 1}}    \\
        \end{align*}
        We know that the time taken for $x$ to reach its first minimum is
        $\pi/\Omega_D$.\\
        We want to approximate $K$ using the binomial approximation, so we
        need to ensure that $|\pi/\log(1/\epsilon)| < 1$ and selecting a
        sufficiently small value of $\epsilon$ we can ensure that and so
        approximate $K$ as
        $$K^2 = \Omega^2\left[\left(\frac{\pi}{\log{1/\epsilon}}\right)^2 + 1\right]^{-1}
            \approx \Omega^2\left[1 - \left(\frac{\pi}{\log{1/\epsilon}}\right)^2\right]$$
        with this approximation we calculate $\Omega_D$ as 
        \begin{align*}
            \Omega_D^2 &= \Omega^2 - K^2 \\
            \Omega_D^2 &= \Omega^2\left[1 - \left[1 - 
            \left(\frac{\pi}{\log{1/\epsilon}}\right)^2\right]\right]\\
            \Omega_D^2 &= \Omega^2\left(\frac{\pi}{\log{1/\epsilon}}\right)^2\\
            \Omega_D &= \frac{\Omega\pi}{\log{1/\epsilon}}
        \end{align*}
        therefore
        $$\frac{\pi}{\Omega_D} = \frac{\log{1/\epsilon}}{\Omega}$$
    \end{proof}
	\begin{proof}{\textbf{5.10}}
        First we need to find the driven response $x^D$. The complex
        counterpart of the equation of motion is
        $$\ddot{x} + \Omega^2\dot{x} = F_0 e^{i\Omega(1+\epsilon)t}$$
        and we seek a solution for this equation of the form
        $x = ce^{i\Omega(1+\epsilon)t}$, substituting we find that
        $$c = \frac{F_0}{\Omega^2(1-(1+\epsilon)^2)} = -\frac{F_0}{\Omega^2\epsilon(2 +\epsilon)}$$
        then the driven response is given by
        $$x^D = \Re\left[-\frac{F_0e^{i\Omega(1+\epsilon)t}}{\Omega^2\epsilon(2 +\epsilon)}\right]
            = -\frac{F_0\cos{(\Omega(1+\epsilon)t)}}{\Omega^2\epsilon(2 +\epsilon)}$$
        To find the complementary function $x^{CF}$, we need to solve
        $$\ddot{x} + \Omega^2\dot{x} = 0$$
        which is going to have a solution of the form
        $$x = A\cos{\Omega t} + B\sin{\Omega t}$$
        then the general solution is going to be
        $$x = A\cos{\Omega t} + B\sin{\Omega t} -
            \frac{F_0\cos{(\Omega(1+\epsilon)t)}}{\Omega^2\epsilon(2 +\epsilon)}$$
        and we can find the values of the constants $A$ and $B$ taking into
        account the initial conditions $x =0$ and $\dot{x} = 0$ when $t=0$, so
        from $x(0) = 0$ we have that
        $$A = \frac{F_0}{\Omega^2\epsilon(2 +\epsilon)}$$
        then differentiating $x$ as
        $$\dot{x} = -A\Omega\sin{\Omega t} + B\Omega\cos{\Omega t} + 
            \frac{F_0(1+\epsilon)}{\Omega\epsilon(2 +\epsilon)}\sin{(\Omega(1+\epsilon)t)}$$
        and computing $\dot{x}(0) = 0$ we have that
        $$B = 0$$
        therefore the general solution is
        $$x = \frac{F_0}{\Omega^2\epsilon(2 +\epsilon)}[cos{(\Omega t)} - \cos{(\Omega(1+\epsilon)t)}]$$
        therefore by the trigonometric identity we have
        $$x = \frac{F_0}{\Omega^2\epsilon(1 +\frac{1}{2}\epsilon)}\sin{(\Omega t(1 +\frac{1}{2}\epsilon))}\sin{(\frac{1}{2}\Omega t\epsilon)}$$    
    \end{proof}
	\begin{proof}{\textbf{5.11}}
        If we assume that the profile of the road grows really slow then we can
        approximate $x$ as $x = ct$ then the displacement of the spring is 
        given by
        $$\delta = y - h(ct)$$
        and the equation of motion can be written as
        $$m\ddot{y} = -\alpha\delta - \beta\dot{\delta}$$
        where $\alpha = m\Omega^2$ is the spring constant and $\beta = 2mK$ is
        the resistance constant, so replacing values we have
        \begin{align*}
            m\ddot{y} &= -m\Omega^2(y - h(ct)) - 2mK(\dot{y} - ch'(ct)) \\
            \ddot{y} + 2K\dot{y} + \Omega^2y &= 2Kch'(ct) + \Omega^2h(ct)
        \end{align*}
        which is the equation we were looking for.
        
        If we suppose now that $h(x) = h_0 \cos(px/c)$ the equation is given by
        $$\ddot{y} + 2K\dot{y} + \Omega^2y = h_0(-2Kp\sin{pt} + \Omega^2\cos{pt})$$
        which can be written as its complex counterpart as
        $$\ddot{y} + 2K\dot{y} + \Omega^2y = h_0(2iKp+\Omega^2)e^{ipt}$$
        and we seek a solution of this equation of the form $y = ce^{ipt}$ then
        by replacing we obtain
        $$-p^2c + 2Kcip + \Omega^2c = h_0(2iKp + \Omega^2)$$
        then $c$ is given by
        $$c = \frac{h_0(2iKp + \Omega^2)}{-p^2 + 2Kip + \Omega^2}$$
        but we want $a$ which is given by $a = |c|$ so
        \begin{align*}
            a &= \frac{h_0|2iKp + \Omega^2|}{|\Omega^2 -p^2 + 2Kip|} \\
              &= \frac{h_0\sqrt{\Omega^4 + (2Kp)^2}}{\sqrt{(\Omega^2-p^2)^2 + (2Kp)^2}}            
        \end{align*}

        If $K = \Omega$ and changing variables taking $u = \Omega^2/p^2$ we
        have that
        \begin{align*}
            \frac{a^2}{h_0^2} &= \frac{\Omega^4+(2\Omega p)^2}{\Omega^4 - 2\Omega^2p^2 + p^4 + 4\Omega^2p^2} \\
             &= \frac{\Omega^2p^2(\Omega^2/p^2 + 2^2)}{\Omega^2p^2(\Omega^2/p^2 + 2 + p^2/\Omega^2)} \\
             &= \frac{u + 4}{u + 2 + 1/u} \\
             &= \frac{u(u + 4)}{u^2 + 2u + 1}
        \end{align*}
        we want to know the maximum value of $a$ then we name $f = a^2/h_0^2$
        and we differentiate the above expression with respect to $u$
        \begin{align*}
            \frac{df}{du} &= \frac{(u^2 + 2u + 1)(u + 4 + u) - (u(u + 4))(2u + 2)}{(u^2 + 2u + 1)^2} \\
                &= \frac{2u^3 + 8u^2 + 10u + 4 - (2u^3 + 10u^2 + 8u)}{(u^2 + 2u + 1)^2} \\
                &= \frac{-2u^2 + 2u + 4}{(u^2 + 2u + 1)^2}            
        \end{align*}
        now we make $df/du = 0$ and solve for $u$
        $$2(u^2 - u - 2) = 0$$
        which roots are $u = 2$ and $u = -1$, and given that $\Omega$ and $p$
        are positive constants then we are interested in $u = 2$. Replacing in
        the equation for $a^2/h_0^2$ we have that
        $$\frac{a^2}{h_0^2} = \frac{12}{9}$$
        then
        $$a \leq \sqrt{\frac{12}{9}}h_0 = \frac{2}{\sqrt{3}}h_0$$
    \end{proof}
\cleardoublepage
    \begin{proof}{\textbf{5.12}}
        The first step is to find the Fourier series of the function $F(t)$. From
        the Fourier series equation, the coefficient $a_n$ is given by
        \begin{align*}
            a_n &= \frac{1}{\pi}\int_{-\pi}^{\pi} F(t) \cos(nt)dt\\
                &= \frac{F_0}{\pi}\int_{-\pi}^{\pi} t\cos(nt)dt \\
                &= \frac{F_0}{\pi}\left[\frac{nt\sin(nt)+\cos(nt)}{n^2}\right]_{-\pi}^{\pi} \\
                &= \frac{F_0}{\pi}\left[\frac{\cos(n\pi)}{n^2} - \frac{\cos(-n\pi)}{n^2}\right] \\
                &= 0
        \end{align*}
        In the same way we calculate $b_n$ as
        \begin{align*}
            b_n &= \frac{1}{\pi}\int_{-\pi}^{\pi} F(t) \sin(nt)dt\\
                &= \frac{F_0}{\pi}\int_{-\pi}^{\pi} t\sin(nt)dt \\
                &= \frac{F_0}{\pi}\left[\frac{\sin(nt)-nt\cos(nt)}{n^2}\right]_{-\pi}^{\pi} \\
                &= \frac{F_0}{\pi}\left[\frac{-\pi\cos(n\pi)}{n} - \frac{\pi\cos(-n\pi)}{n}\right] \\
                &= -\frac{2F_0\cos(n\pi)}{n} = -\frac{2F_0(-1)^n}{n}
        \end{align*}
        Hence the Fourier series of the function $F(t)$ is
        $$F(t) = \sum^{\infty}_{n=1}\frac{-2F_0(-1)^n}{n} \sin(nt)$$
        The next step is to find the driven response to the force
        $mb_n\sin(nt)$, that is, the particular integral of the equation
        $$\frac{d^2x}{dt^2} + 2K\frac{dx}{dt} + \Omega^2x = b_n\sin(nt) \quad\quad (1)$$
        the complex counterpart of this equation is
        $$\frac{d^2x}{dt^2} + 2K\frac{dx}{dt} + \Omega^2x = b_ne^{int}$$
        and we seek for a solution of the form $x = ce^{int}$ then by replacing
        in the equation we get that
        $$-cn^2 + 2iKcn + \Omega^2c = b_n$$
        so the complex amplitude is 
        $$c = \frac{b_n}{\Omega^2 - n^2 + 2iKn}$$
        then the solution for the real equation $(1)$ is given by
        $$\Im\left[\frac{b_ne^{int}}{\Omega^2 - n^2 + 2iKn}\right] = 
            b_n\left(\frac{(\Omega^2-n^2)\sin(nt) + 2Kn\cos(nt)}{(\Omega^2-n^2)^2+4K^2n^2}\right)$$
        Finally we add together these separate responses to find the driven response
        $$x = -2F_0\sum_{n=1}^{\infty}\frac{(-1)^n}{n}\left(\frac{(\Omega^2-n^2)\sin(nt) + 2Kn\cos(nt)}{(\Omega^2-n^2)^2+4K^2n^2}\right)$$
    \end{proof}
    \begin{proof}{\textbf{5.15}}
        Given that the oscillator is partially damped the equations of motion
        are the following
        \begin{align}
            \ddot{x} + 2K\dot{x} + \Omega^2x = 0 \quad&\text{when}\quad x>0\\
            \ddot{x} + \Omega^2x = 0 \quad&\text{when}\quad x<0
        \end{align}
        and we know that the solutions to these equations are
        \begin{align}
            x_{+} = e^{-Kt}(A\cos{\Omega_Dt} + B\sin{\Omega_Dt}) \\
            x_{-} = A\cos{\Omega t} + B\sin{\Omega t}
        \end{align}

        then $(3)$ has a period of $\tau = 2\pi/\Omega_D$ and $(4)$ has a
        period of $\tau = 2\pi/\Omega$. These two solutions must match at $x=0$
        so the total period should be half the period of $(3)$ and half the
        period of $(4)$ then
        $$\tau = \frac{\frac{2\pi}{\Omega_D}}{2} + \frac{\frac{2\pi}{\Omega}}{2}
            = \pi\left(\frac{1}{\Omega_D} + \frac{1}{\Omega}\right)$$

        The maximum values of $x$ are going to happen on the positive side 
        of the oscillation i.e. on the side of $x_{+}$. If the first maximum
        value happen at $t = 0$ then
        $$x = e^{-K0}(A_1\cos{\Omega_D0} + B_1\sin{\Omega_D0}) = A_1$$
        the next maximum is going to happen at $t=2\tau$ but we first need to
        determine the constants for the half negative period where $x$ is given
        by
        $$x = A_2\cos{\Omega t} + B_2\sin{\Omega t}$$
        Since this is a tedious algebraic calculation we are going to say that
        the constants for the half negative period can be determined in terms
        of $A_1$, $B_1$ so $A_2 = f(A_1, B_1, B_2)$ and $B_2 = g(A_1, B_1, A_2)$
        therefore the values of the constants for the half positive period are
        also functions of $A_1$ and $B_1$, so we are going to write the ratio
        between maximums just as
        \begin{align*}
            \text{maximums ratio} &= \frac{e^{-2K\tau}(A_3\cos(\Omega_D2\tau) + B_3\sin(\Omega_D2\tau))}{e^{-K0}(A_1\cos{\Omega_D0} + B_1\sin{\Omega_D0})} \\
                &= \frac{e^{-2K\tau}(A_3\cos(\Omega_D2\tau) + B_3\sin(\Omega_D2\tau))}{A_1}
        \end{align*}
    \end{proof}
    \begin{proof}{\textbf{5.18}}
        In this case, we define $x_1$ and $x_2$ as the displacements
        perpendicular to the downward vertical that passes through $O$ for the
        particles $P$ and $Q$ respectively. Also, we name $y_1$ and $y_2$ to 
        the displacements in the downward vertical direction for each particle.
        Then the equations of motion for particle $P$ can be written as
        \begin{align*}
            3m\ddot{x_1} &= T_1\sin\phi -T_0\sin\theta  \\
            3m\ddot{y_1} &= T_1\cos\phi + 3mg - T_0\cos\theta            
        \end{align*}
        and for particle $Q$ as
        \begin{align*}
            m\ddot{x_2} &= - T_1\sin\phi \\
            m\ddot{y_2} &= mg - T_1\cos\phi
        \end{align*}
        If the displacements of $x_1$ and $x_2$ are small then we could say
        that $x_1/a \approx \theta$, $x_2/a \approx \theta + \phi$,
        $\sin\theta \approx \theta$ and that $\sin\phi \approx \phi$ also
        we can neglect the displacement in the $y_1$ and $y_2$ direction so
        the cosines of the angles are approximated to $1$ then we have for the
        particle $P$ that
        \setcounter{equation}{0}
        \begin{align}
            3ma\ddot{\theta} &= T_1\phi - T_0\theta \\
            0 &= T_1 + 3mg - T_0            
        \end{align}
        and for the particle $Q$ we have that
        \begin{align}
            ma(\ddot{\theta} + \ddot{\phi}) &= -T_1\phi \\
            0 &= mg - T_1
        \end{align}
        from $(4)$ we have that $T_1 = mg$ and then replacing that in $(2)$ we
        have that $T_0 = 4mg$ then we get the following equations
        \begin{align*}
            3\ddot{\theta} + 4n^2\theta - n^2\phi &= 0\\
            \ddot{\theta} + \ddot{\phi} + n^2\phi &= 0 
        \end{align*}
        now replacing the value of $\phi$ from the last equation in the first
        one we get the following set of equations
        \begin{align}
            4\ddot{\theta} + \ddot{\phi} + 4n^2\theta &= 0\\
            \ddot{\theta} + \ddot{\phi} + n^2\phi &= 0 
        \end{align}
        which are the equations we wanted.
\cleardoublepage
        The equations will have normal mode solutions of the form
        \begin{align*}
            \theta &= A\cos(\omega t - \gamma)\\
            \phi &= B\cos(\omega t - \gamma)
        \end{align*}
        then replacing the normal mode equations in the equations $(5)$ and
        $(6)$ we obtain that
        \begin{align*}
            -4\omega^2A\cos(\omega t - \gamma) -\omega^2B\cos(\omega t - \gamma) + 4n^2A\cos(\omega t - \gamma) &= 0 \\
            -\omega^2A\cos(\omega t - \gamma) -\omega^2B\cos(\omega t - \gamma) + n^2B\cos(\omega t - \gamma) &= 0
        \end{align*}
        which simplifying gives
        \begin{align}
            4(n^2-\omega^2)A -\omega^2B &= 0 \\
            -\omega^2A + (n^2-\omega^2)B &= 0
        \end{align}
        we require the equations to have a non-trivial solution for $A$ and $B$.
        So the determinant of the system of equations should be zero, that is
        \begin{align*} 
            \begin{vmatrix}
            4(n^2-\omega^2) & -\omega^2 \\
            -\omega^2 & (n^2-\omega^2) 
            \end{vmatrix}
            &= 0
        \end{align*}
        On simplification, this gives the condition
        $$3\omega^4 -8n^2\omega^2 + 4n^4 = 0$$
        a quadratic equation in the variable $\omega^2$ which solutions are
        $$\omega_1^2 = 2n^2 \quad\quad\quad \omega_2^2 =\frac{2}{3}n^2$$
        then the normal frequencies are $\omega_1 = \sqrt{2}n$ and
        $\omega_2 = \sqrt{2/3}n$

        Now we replace the values of the normal frequencies in the equations
        $(7)$ and $(8)$. We have for $\omega_1$ that $B=-2A$ then we have thus
        found a family of non-trivial solutions $A=\delta$ and $B=-2\delta$,
        where $\delta$ can take any (non-zero) value. The normal mode therefore
        has the form
        \begin{align*}
            \theta &= \delta\cos(\omega_1 t - \gamma_1)\\
            \phi &= -2\delta\cos(\omega_1 t - \gamma_1)
        \end{align*}
        In the same way for $\omega_2$ we have that $B = 2A$ then $A=\delta$
        and $B = 2\delta$. The normal mode therefore has the form
        \begin{align*}
            \theta &= \delta\cos(\omega_2 t - \gamma_2)\\
            \phi &= 2\delta\cos(\omega_2 t - \gamma_2)
        \end{align*}
        The general motion is now the sum of the first normal mode (with
        amplitude factor $\delta_1$ and phase factor $\gamma_1$) and the second
        normal mode (with amplitude factor $\delta_2$ and phase factor
        $\gamma_2$).
        \begin{align*}
            \theta &= \delta_1\cos(\omega_1 t - \gamma_1) + \delta_2\cos(\omega_2 t - \gamma_2)\\
            \phi &= -2\delta_1\cos(\omega_1 t - \gamma_1) + 2\delta_2\cos(\omega_2 t - \gamma_2)
        \end{align*}
        For this system $\tau_1/\tau_2 = \omega_2/\omega_1 = 1/\sqrt{3}$ so
        the general motion is not periodic.
    \end{proof}

\end{document}






















