\documentclass[11pt]{article}
\usepackage{amssymb}
\usepackage{amsthm}
\usepackage{enumitem}
\usepackage{amsmath}
\usepackage{bm}
\usepackage{adjustbox}
\usepackage{mathrsfs}
\usepackage{graphicx}
\usepackage{siunitx}
\usepackage[mathscr]{euscript}

\title{\textbf{Solved selected problems of Classical Mechanics - Gregory}}
\author{Franco Zacco}
\date{}

\addtolength{\topmargin}{-3cm}
\addtolength{\textheight}{3cm}

\newcommand{\hatr}{\bm{\hat{r}}}
\newcommand{\hatx}{\bm{\hat{x}}}
\newcommand{\haty}{\bm{\hat{y}}}
\newcommand{\hatz}{\bm{\hat{z}}}
\newcommand{\hatth}{\bm{\hat{\theta}}}
\newcommand{\hatphi}{\bm{\hat{\phi}}}
\newcommand{\hatrho}{\bm{\hat{\rho}}}
\theoremstyle{definition}
\newtheorem*{solution*}{Solution}

\begin{document}
\maketitle
\thispagestyle{empty}

\section*{Chapter 6 - Energy conservation}

	\begin{proof}{\textbf{6.6}}
        Given that the potential energy is $V = V_0(x/b)^4$ then the potential energy is
        a U-shaped curve where the lowest point is at $x=0$ where $V=0$ and because of
        the energy conservation $V$ cannot go above a threshold given by $E$, therefore
        the only possible motion for a particle is an oscillation around $x=0$.

        Let us suppose now suppose that the oscillation has an amplitude of $a$ so when
        $x=a$ then $v=0$ and by the energy conservation formula with this initial
        conditions  we have that
        $$E = V_0 (\frac{a}{b})^4$$
        Then from the energy conservation formula we have that
        \begin{align*}
            \frac{1}{2}m(\frac{dx}{dt})^2 +V_0(\frac{x}{b})^4 &= V_0 (\frac{a}{b})^4 \\
            \frac{dx}{dt} &= \sqrt{\frac{2}{m}V_0((\frac{a}{b})^4 - (\frac{x}{b})^4)}\\
            \frac{dx}{dt} &= \sqrt{\frac{2}{m}V_0\frac{a^4-x^4}{b^4}}
        \end{align*}
        And from here we can determine $\tau$ solving the ODE integrating one side from
        0 to $a$ which is the aplitude and the other side from 0 to $\tau/4$ because
        we are taking into account a quarter of a period, then
        \begin{align*}
            \int_0^{\tau/4} dt &= \int_{0}^{a} \frac{dx}{\sqrt{\frac{2}{m}V_0\frac{a^4-x^4}{b^4}}}\\
            \tau &= \frac{4}{\sqrt{2}}\sqrt{\frac{m}{V_0}}b^2
                    \int_{0}^{a} \frac{dx}{\sqrt{a^4-x^4}}\\
                &= \frac{4}{\sqrt{2}}\sqrt{\frac{m}{V_0}}b^2
                \int_{0}^{a} \frac{dx}{\sqrt{a^4-x^4}}
        \end{align*}
        Let us now do the following variable replacement, let us define $\xi = x/a$
        then changing the limits of integration correspondingly we get that 
        \begin{align*}
            \tau &= 2\sqrt{2}\sqrt{\frac{m}{V_0}}b^2
                \int_{0}^{1} \frac{ad\xi}{\sqrt{a^4-(a\xi)^4}}\\
                &= 2\sqrt{2}\sqrt{\frac{m}{V_0}}b^2
                \int_{0}^{1} \frac{ad\xi}{\sqrt{a^4(1-\xi^4)}}\\
                &= 2\sqrt{2}\sqrt{\frac{m}{V_0}}\frac{b^2}{a}
                \int_{0}^{1} \frac{d\xi}{\sqrt{1-\xi^4}}
        \end{align*}
    \end{proof}
    \begin{proof}{\textbf{6.7}}
        Given that the particle is moving towards the origin with speed $u$ then we can
        determine the total energy $E$ at this moment as
        $$E = \frac{1}{2}mu^2$$
        Then when the particle is at the origin it experiences a force $F = -Kx^2$ so
        by integrating this equation we can get the potential energy as
        \begin{align*}
            V &= -\int_0^x -K{x'}^2 dx'\\
              &= K\int_0^x {x'}^2 dx'\\
              &= K\left[\frac{x^3}{3} -0\right] = K\frac{x^3}{3}
        \end{align*}
        Where $x$ is how far the particle is going to get given that the force that the
        particle is experiencing is decelerating it.
        Let us now write the corresponding energy conservation equation at this moment
        as
        \begin{align*}
            \frac{1}{2}mv^2 + K\frac{x^3}{3} &= \frac{1}{2}mu^2\\
            K\frac{x^3}{3} &= \frac{1}{2}m(u^2-v^2)\\
            x &= \sqrt[3]{\frac{3}{2}\frac{m}{K}(u^2-v^2)}
        \end{align*} 
        So at $v=0$ we get the furthest point that the particle is going to get.
        $$x = \sqrt[3]{\frac{3}{2}\frac{m}{K}u^2}$$ 
    \end{proof}
\cleardoublepage
    \begin{proof}{\textbf{6.8}}
        The potential energy $V(x)$ for the force $F(x)$ we have is given by
        \begin{align*}
            V(x) &= - \int - \frac{2mMGx}{(a^2 + x^2)^{3/2}} dx\\
                 &= 2mMG \int \frac{x}{(a^2 + x^2)^{3/2}} dx\\
                 &= 2mMG \left[ -\frac{1}{\sqrt{a^2+x^2}}\right]
        \end{align*}
        During the period from $x=3a/4$ to $x=0$ the energy conservation equation is
        given by
        $$\frac{1}{2}mv^2 - \frac{2mMG}{\sqrt{a^2+x^2}} = E$$
        and using the initial conditions $v=0$ and $x=3a/4$ we get that
        \begin{align*}
            E &= 0 - \frac{2mMG}{\sqrt{a^2+(3a/4)^2}}\\
              &= - \frac{2mMG}{\sqrt{\frac{25}{16}a^2}}\\
              &= - \frac{8mMG}{5a}
        \end{align*}
        Therefore the complete energy conservation equation is
        $$\frac{1}{2}mv^2 - \frac{2mMG}{\sqrt{a^2+x^2}} = - \frac{8mMG}{5a}$$
        So when the particle reaches the origin the force in the $x$ direction is 0
        therefore the maximum speed is reached when $x=0$, then with this in mind 
        the energy conservation equation turns into
        \begin{align*}
            \frac{1}{2}mv^2 &= \frac{2mMG}{\sqrt{a^2}} - \frac{8mMG}{5a}\\
            \frac{1}{2}mv^2 &= \frac{2mMG}{5a}\\
                          v &= \sqrt{\frac{4MG}{5a}}
        \end{align*}
    \end{proof}
\cleardoublepage
    \begin{proof}{\textbf{6.9}}
        The potential energy $V(x)$ for the force $F(x)$ we have is given by
        \begin{align*}
            V(x) &= - \int - \frac{2mMG}{a^2}\left[1 - \frac{z}{\sqrt{a^2 + z^2}}\right] dz\\
                 &= \frac{2mMG}{a^2} \int 1 - \frac{z}{\sqrt{a^2 + z^2}} dz\\
                 &= \frac{2mMG}{a^2} \left[z - \int \frac{z}{\sqrt{a^2 + z^2}} dz\right]\\
                 &= \frac{2mMG}{a^2} \left[z - \sqrt{a^2+z^2}\right]
        \end{align*}
        During the period from $z=4a/3$ to $z=0$ the energy conservation equation is
        given by
        $$\frac{1}{2}mv^2 + \frac{2mMG}{a^2} \left[z - \sqrt{a^2+z^2}\right] = E$$
        and using the initial conditions $v=0$ and $z=4a/3$ we get that
        \begin{align*}
            E &= 0 + \frac{2mMG}{a^2} \left[\frac{4}{3}a - \sqrt{a^2+(\frac{4}{3}a)^2}\right]\\
              &= \frac{2mMG}{a^2} \left[\frac{4}{3}a - a\sqrt{\frac{25}{9}}\right]\\
              &= \frac{2mMG}{a}\left[\frac{4-5}{3}\right]\\
              &= -\frac{2mMG}{3a}
        \end{align*}
        Therefore the complete energy conservation equation is
        $$\frac{1}{2}mv^2 + \frac{2mMG}{a^2} \left[z - \sqrt{a^2+z^2}\right] = -\frac{2mMG}{3a}$$
        So when the particle reaches the origin (hitting the disk) the energy
        conservation equation turns into
        \begin{align*}
            \frac{1}{2}mv^2 &= \frac{2mMG}{a} - \frac{2mMG}{3a}\\
            \frac{1}{2}mv^2 &= \frac{4mMG}{3a}\\
                          v &= \sqrt{\frac{8MG}{3a}}
        \end{align*}
    \end{proof} 
    \begin{proof}{\textbf{6.17}}
        The forces acting on P are uniform gravity $mg$ and the constraint force N which
        is the normal reaction of the smooth wire. Since $N$ is perpendicular to
        $v$ it follows that $N$ does no work. Then the energy conservation equation is
        given by
        $$\frac{1}{2}mv^2 + mgz = E$$
        So by plugging the intial contidions $v=0$ when $z=2\pi b$ we have that
        $E= 2\pi mgb$.
        Then the energy conservation equation becomes
        $$\frac{1}{2}v^2 + gz = 2\pi gb$$
        And the velocity of the particle is given by
        $$v = \sqrt{2g(2\pi b - z)}$$
        Then to get the arrival velocity we plug $z=0$ and we get that
        $$v = 2\sqrt{g\pi b}$$
        Let us now calculate the time taken to reach the point $(a, 0, 0)$.
        From the equations we have for the position of the particle we can calculate the
        velocity at any point on the helix as follows. The components of the velocity
        are given by
        \begin{align*}
            \dot{x} &= -a\dot{\theta} \sin{(\theta)}\\
            \dot{y} &= a\dot{\theta} \cos{(\theta)}\\
            \dot{z} &= b\dot{\theta}
        \end{align*}
        Then the magnitude of the velocity is
        \begin{align*}
            v^2 &= \dot{x}^2 + \dot{y}^2 + \dot{z}^2\\
                &= (-a\dot{\theta} \sin^2{(\theta)}) + (a\dot{\theta} \cos^2{(\theta)}) + (b\dot{\theta})^2\\
                &= a^2\dot{\theta}^2(\sin^2{(\theta)} + \cos^2{(\theta)}) + b^2\dot{\theta}^2\\
                &= \dot{\theta}^2(a^2 + b^2)
        \end{align*}
        So replacing this value in the equation we have for $v$ (that we got from the
        energy conservation formula) we have that
        \begin{align*}
            v^2 &= 2g(2\pi b - b\theta)\\
            \dot{\theta}^2(a^2 + b^2) &= 2gb(2\pi - \theta)\\
            \dot{\theta} \sqrt{a^2 + b^2} &= \sqrt{2gb}\sqrt{2\pi - \theta}\\
            \frac{d\theta}{dt}\frac{\sqrt{a^2 + b^2}}{\sqrt{2\pi - \theta}} &= \sqrt{2gb}\\
            \sqrt{a^2 + b^2}\int_0^{2\pi} \frac{d\theta}{\sqrt{2\pi - \theta}} &= \sqrt{2gb}\int_0^t dt'\\
            \sqrt{a^2 + b^2}\left[-2\sqrt{2\pi-\theta}\right]_0^{2\pi} &= t\sqrt{2gb}\\
            t &= 2\sqrt{\frac{\pi(a^2 + b^2)}{gb}}
        \end{align*}
    \end{proof}
    \begin{proof}{\textbf{6.18}}
        Let us first compute the energy conservation equation as follows
        $$\frac{1}{2}mv^2 + mgz = E$$
        If we assume that the ball starts from rest when $x = a$ then by applying this
        initial conditions we have that
        \begin{align*}
            E &= \frac{mga^2}{2b}
        \end{align*}
        Where we used that $z = x^2/2b$. Then the complete energy conservation equation
        is given by
        $$\frac{1}{2}mv^2 + mgz = \frac{mga^2}{2b}$$
        Now let us calculate the value of $v$ as follows
        \begin{align*}
            v^2 &= \dot{z}^2 + \dot{x}^2\\
                &= \left(\frac{x\dot{x}}{b}\right)^2 + \dot{x}^2\\
                &= \dot{x}^2\left(\left(\frac{x}{b}\right)^2 + 1\right)
        \end{align*}
        Then from the energy conservation equation we have that
        \begin{align*}
            \frac{1}{2}\dot{x}^2\left(\left(\frac{x}{b}\right)^2 + 1\right) + \frac{gx^2}{2b} &= \frac{ga^2}{2b}\\
            \frac{1}{2}\dot{x}^2\left(\left(\frac{x}{b}\right)^2 + 1 \right) &= \frac{ga^2}{2b} - \frac{gx^2}{2b}\\
            \dot{x}^2\left(\left(\frac{x}{b}\right)^2 + 1 \right) &= \frac{g}{b}(a^2-x^2)\\
            \dot{x}^2 &= \frac{g}{b}\frac{(a^2-x^2)}{\left(\frac{x}{b}\right)^2 + 1}\\
            \dot{x}^2 &= \frac{g}{b}\frac{b^2(a^2-x^2)}{x^2 + b^2}\\
            \dot{x} &= \sqrt{bg}\sqrt{\frac{a^2-x^2}{x^2 + b^2}}\\
            \int_0^a \frac{dx}{\sqrt{\frac{a^2-x^2}{x^2 + b^2}}} &= \sqrt{bg}\int_0^{\tau/4} dt\\
            \tau &= \frac{4}{\sqrt{bg}} \int_0^a \sqrt{\frac{x^2+b^2}{a^2 - x^2}}dx
        \end{align*}
        Now let us do the replacement $x = a \sin \psi$ then
        \begin{align*}
            \tau &= \frac{4}{\sqrt{bg}} \int_0^{\pi/2} \sqrt{\frac{a^2\sin^2\psi+b^2}{a^2 - a^2\sin^2\psi}}a\cos\psi d\psi\\
                 &= \frac{4}{\sqrt{bg}} \int_0^{\pi/2} \sqrt{\frac{
                    b^2(\frac{a^2}{b^2}\sin^2\psi+1)}{a^2(1 - \sin^2\psi)}}a\cos\psi d\psi\\
                &= 4\sqrt{\frac{b}{g}} \int_0^{\pi/2} \sqrt{\frac{
                    (\frac{a^2}{b^2}\sin^2\psi+1)}{a^2 \cos^2\psi}}a\cos\psi d\psi\\
                &= 4\sqrt{\frac{b}{g}} \int_0^{\pi/2} \sqrt{\frac{a^2}{b^2}\sin^2\psi+1}~ d\psi
        \end{align*}
        When the ratio $a/b$ is small the integrand can be approximated by its taylor
        series as
        \begin{align*}
            \sqrt{\frac{a^2}{b^2}\sin^2\psi+1} &\approx 1 + \frac{a^2}{b^2}\frac{\sin^2\psi}{2} - O(\frac{a^4}{b^4}) 
        \end{align*}
        Then our approximated integral with two terms becomes
        \begin{align*}
            \tau &= 4\sqrt{\frac{b}{g}} \int_0^{\pi/2} 1 + \frac{a^2}{b^2}\frac{\sin^2\psi}{2} ~d\psi\\
                &= 4\sqrt{\frac{b}{g}} \left[\frac{\pi}{2} + \left[\frac{a^2}{2b^2}\frac{1}{2}(\psi - \sin\psi \cos \psi)\right]_0^{\pi/2}\right]\\
                &= 4\sqrt{\frac{b}{g}} \left[\frac{\pi}{2} + \frac{a^2}{2b^2}\frac{\pi}{4}\right]\\
                &= 2\pi\sqrt{\frac{b}{g}} \left[1 + \frac{a^2}{4b^2}\right]
        \end{align*}
    \end{proof}
\cleardoublepage
    \begin{proof}{\textbf{6.19}}
        From the equations we have for the coordinates of the particle we can compute $v$
        as follows
        \begin{align*}
            v^2 &= \dot{x}^2 + \dot{z}^2\\
                &= c^2\dot{\theta}^2(1 + \cos\theta)^2 + c^2\dot{\theta}^2\sin^2\theta\\
                &= c^2\dot{\theta}^2(1 + 2\cos\theta + \cos^2\theta + \sin^2\theta)\\
                &= 2c^2\dot{\theta}^2(1 + \cos\theta)
        \end{align*}
        Now from the energy conservation equation, we have to do the corresponding
        variable replacements
        \begin{align*}
            \frac{1}{2}mv^2 + mgz &= E\\
            \frac{1}{2}m(2c^2\dot{\theta}^2(1 + \cos\theta)) + mgc(1-\cos\theta) &= E\\
            mc^2(\dot{\theta}^2(1 + \cos\theta) + \frac{g}{c}(1-\cos\theta)) &= E\\
            \dot{\theta}^2(1 + \cos\theta) + \frac{g}{c}(1-\cos\theta) &= \frac{E}{mc^2} = \text{constant}
        \end{align*}
        Therefore we have the equation we wanted using the fact that the total energy
        $E$ is constant and $m$ and $c$ are positive constants.

        If we define $u = \sin{\frac{\theta}{2}}$ then we have that
        \begin{align*}
            \dot{u}^2 &= \frac{\dot{\theta}^2}{4}\cos^2{\frac{\theta}{2}}\\
            8\dot{u}^2 &= 2\dot{\theta}^2\cos^2{\frac{\theta}{2}}
        \end{align*}
        Now from the energy conservation equation we want to express both terms as a
        function of $u$ then
        \begin{equation*}
            \begin{split}
                1- \cos\theta &= 2\sin^2\frac{\theta}{2}\\
                              &= 2u^2
            \end{split}
            \quad\quad\quad
            \begin{split}
                \dot{\theta}^2(1 + \cos\theta) &= 2\dot{\theta}^2\cos^2\frac{\theta}{2}\\
                    &= 8\dot{u}^2 
            \end{split}
        \end{equation*}
        Then our energy conservation equation becomes
        $$8\dot{u}^2 + \frac{2g}{c}u^2 = \text{constant}$$
        Now we compute one more derivative of this expression and we have that
        \begin{align*}
            8(2\dot{u}\ddot{u}) + \frac{2g}{c}(2u\dot{u}) &= 0\\
            16\dot{u}(\ddot{u} + \frac{g}{4c}u) &= 0\\
            \ddot{u} + \frac{g}{4c}u &= 0
        \end{align*}
        
        The equation in terms of $u$ has the form of a Simple Harmonic Motion so we can
        calculate the period $T$ of this motion as follows
        $$T = 2\pi \sqrt{\frac{4c}{g}} = 4\pi \sqrt{\frac{c}{g}}$$
    \end{proof}
    \begin{proof}{\textbf{6.22}}
        Given that the polished surface in hemisphere shape generates a reactive force
        perpendicular to the velocity $v$ then it does no work and we can use the
        conventional energy conservation equation as follows
        \begin{align*}
            \frac{1}{2}mv^2 + mgz = E
        \end{align*}
        Let us define now $\theta$ as the angle between the perpendicular line that
        passes through the center of the hemisphere O and the
        line that passes through the center O and follows Vita along her slide down
        (point P).
        Then the height $z$ at which Vita is going to be at any time is
        given by $z = a\cos\theta$ where $a$ is the radius of the hemisphere.
        So the energy equation becomes
        $$\frac{1}{2}mv^2 + mga\cos\theta = E$$
        And by filling the initial conditions $v=0$ and $\theta=0$ we have that
        $$E = mga \quad\quad\text{then}\quad\quad \frac{1}{2}mv^2 + mga\cos\theta = mga$$
        So the velocity is given by
        $$v^2 = 2ga(1 - \cos\theta)$$
        
        Now let us calculate the reaction of the surface using the Second Law as follows
        \begin{align*}
            mg\cos\theta - N &= m\frac{v^2}{a}\\
            N &= m(g\cos\theta - \frac{v^2}{a})\\
            N &= m(g\cos\theta - \frac{2ga(1 - \cos\theta)}{a})\\ 
            N &= mg(\cos\theta - 2(1 - \cos\theta))\\
            N &= mg(3\cos\theta - 2)
        \end{align*}
        Where we used that $mg\cos\theta$ is the component of the weight perpendicular
        to the velocity and in the direction of $N$ and $\frac{v^2}{a}$ is the
        acceleration.
        When $N=0$ this implies that Vita won't be sliding anymore and would be falling
        so at this point she starts to fall from the hemisphere, then
        \begin{align*}
            0 &= 3\cos\theta - 2\\
            \theta &= \cos^{-1} \frac{2}{3}
        \end{align*}
        And this is going to be the angle formed between an horizontal line (x-axis)
        passing through P and and the velocity vector. At this point the motion is
        the same as an object thrown from a cliff with a specific angle and initial
        velocity, let us define the z-axis with a positive direction downward and the
        x-axis with a positive direction to the right. The initial velocity is therefore
        given by
        \begin{align*}
            v &= \sqrt{2ga(1 - \cos\theta)}\\
                &= \sqrt{2ga(1 - \frac{2}{3})}\\
                &= \sqrt{\frac{2}{3}ga} = 16.16 m/s
        \end{align*} 
        And the $z$ equation is given by
        \begin{align*}
            z &= 0 + v_{0z}t + \frac{1}{2}gt^2\\
            a\cos\theta &= vt\sin\theta + \frac{1}{2}gt^2\\
            0 &= \frac{1}{2}gt^2 + vt\sin\theta - a\cos\theta
        \end{align*}
        Plugging the values $g=9.8 m/s^2$, $a=40m$, $v=16.16m/s$ and the value we got for
        $\theta = \cos^{-1}\frac{2}{3}$ we have that $t_1 = -2.41$ and $t_2 = 2.25$ where
        we keep $t = 2.25 s$ since the other value doesn't make sense physically.
        Finally the distance where Vita is going to land can be calculated as
        \begin{align*}
            x &= v_{0x}t\\
              &= vt\cos\theta 
        \end{align*}
        Where plugging the values we have we get that $x= 24.24m$
    \end{proof}
\cleardoublepage
    \begin{proof}{\textbf{6.23}}
        When the string hits the peg the movement is transformed into a pendulum motion
        of a string of length $a$, and at some point of the movement the ball will stop
        feeling the tension of the string. At this moment the string will have an angle
        $\beta$ with the vertical line as shown
        \includegraphics[width=\textwidth]{"6.23_graph.jpg"}
        Then the Second Law equation of motion for the centripetal force is given by
        \begin{align*}
            mg\cos\beta &= \frac{mu^2}{a}\\ 
            u^2 &= ag\cos\beta
        \end{align*}
        So at this point the movement is a projectile movement with a specific initial
        velocity $u$ and angle $\beta$ and the trajectory of the ball is given by
        \begin{align*}
            z &= \tan{\beta}x - \frac{g}{2u^2\cos^2\beta}x^2\\
              &= \tan{\beta}x - \frac{1}{2a\cos^3\beta}x^2
        \end{align*}
        Where we replaced the value we have for $u$. From here we want that when
        $x = a \sin\beta$ then $z = -a\cos\beta$ so the ball hits the peg so by
        replacing the values we have that
        \begin{align*}
            -a\cos\beta &= a\frac{\sin^2\beta}{\cos\beta} - \frac{a\sin^2{\beta}}{2\cos^3\beta}\\
            -\cos^2\beta &= \sin^2\beta - \frac{\sin^2{\beta}}{2\cos^2\beta}\\
            0 &= 1 - \frac{\tan^2{\beta}}{2}\\
            \tan{\beta} &= 2\\
            \beta &= \tan^{-1}{\sqrt{2}}
        \end{align*}
        Then from the velocity formula we have that
        $$u^2 = ag\cos{(\tan{\sqrt{2}})} = \frac{ag}{\sqrt{3}}$$
        
        Now since we know that the energy is conserved in the system given that the
        string is inextensible and does no work we can compute the angle $\alpha$ by
        the energy conservation equation where we know that $E = 2amg(1-\cos\alpha)$
        at the begining of the movement because the movement starts from rest and 
        we consider that the highest the ball could be is $2a$ (assuming the $z$-axis
        has its positive direction upwards) then
        \begin{align*}
            g2a(1-\cos\alpha) &= \frac{1}{2}v^2 + gz\\
            g2a(1-\cos\alpha) &= \frac{1}{2}\frac{ag}{\sqrt{3}} + ga(1 + \cos\beta)\\
            2(1-\cos\alpha) &= \frac{1}{2}\frac{1}{\sqrt{3}} + 1 + \frac{1}{\sqrt{3}}\\
            1-\cos\alpha &= \frac{1}{2} + \frac{1}{\sqrt{3}}\frac{3}{4}\\
            \cos\alpha &= 1 - \frac{1}{2} - \frac{\sqrt{3}}{4}\\
            \alpha &= \cos^{-1}(1 - \frac{2 + \sqrt{3}}{4}) = 86,16^{\circ}\\
        \end{align*} 
    \end{proof}



\end{document}






















