\documentclass[11pt]{article}
\usepackage{amssymb}
\usepackage{amsthm}
\usepackage{enumitem}
\usepackage{amsmath}
\usepackage{bm}
\usepackage{adjustbox}
\usepackage{mathrsfs}
\usepackage{graphicx}
\usepackage{siunitx}
\usepackage[mathscr]{euscript}

\title{\textbf{Solved selected problems of Classical Mechanics - Gregory}}
\author{Franco Zacco}
\date{}

\addtolength{\topmargin}{-3cm}
\addtolength{\textheight}{3cm}

\newcommand{\hatr}{\bm{\hat{r}}}
\newcommand{\hatx}{\bm{\hat{x}}}
\newcommand{\haty}{\bm{\hat{y}}}
\newcommand{\hatz}{\bm{\hat{z}}}
\newcommand{\hatth}{\bm{\hat{\theta}}}
\newcommand{\hatphi}{\bm{\hat{\phi}}}
\newcommand{\hatrho}{\bm{\hat{\rho}}}
\theoremstyle{definition}
\newtheorem*{solution*}{Solution}
\renewcommand*{\proofname}{Solution}

\begin{document}
\maketitle
\thispagestyle{empty}

\section*{Chapter 4 - Problems in Particle Dynamics}

	\begin{proof}{\textbf{4.1}}
        The blocks are being towed at constant speed, then the acceleration
        is 0 and since both masses are identical the frictional forces ($F$) are
        identical too then we have the following two equations
        $$0 = T_0 - T - F \quad\quad\quad 0 = T - F$$
        $$T = T_0 - F \quad\quad\quad T = F$$
        the first equation correspond to the block that is being towed and the
        second equation correspond to the other block. Then if we sum both
        equations we get that
        $$2T = T_0 - F + F$$
        therefore
        $$T = \frac{T_0}{2}$$
        but also by subtracting both equations we have that
        $$F = \frac{T_0}{2}$$
        If now the tension of the towed block changes to $4T_0$ instantaneously
        we have the following changes in the equations
        $$Ma = 4T_0 - T - F \quad\quad\quad Ma = T - F$$
        by subtracting both equations we get that $0 = 4T_0 - T - F - T + F$
        so
        $$T = 2T_0$$
        also if we consider that $F$ does not changes at that moment then from
        the second equation and doing the needed replacements we have that
        $$a = \frac{T-F}{M} = \frac{2T_0 - T_0/2}{M} = \frac{3T_0}{2M}$$
    \end{proof}
	\begin{proof}{\textbf{4.3}}
        For this problem we can replace the spheres by particles which are
        $1m$ apart. We want to measure the time it takes for them to move
        $3cm$.
        The acceleration in this case is not constant since it grows as the
        spheres approach, so we can calculate the initial acceleration as
        $$a = \frac{mG}{R^2} = \frac{5000 \cdot 6.67 \times 10^{-11}}{1^2} = 3.335 \times 10^{-7}$$
        and from $x = 1/2at^2$ we can calculate the collision time (an
        approximation) as if the acceleration were constant, then
        $$t = \sqrt{\frac{2x}{a}} = \sqrt{\frac{2 \cdot 0.03}{3.335 \times 10^{-7}}} = 424.15 \si{s}$$
        Since the accelearation grows as the spheres approach, then the time is
        going to decrease because of the inverse depencence with the
        acceleration, therefore the real collision time is even less than the
        one provided, i.e. the approximated time is an upper bound.
    \end{proof}
    \begin{proof}{\textbf{4.4}}
        For this problem we have the following equations already derived
        $$ma = mg\sin\alpha - F \quad\quad 0 = N - mg\cos\alpha$$
        so $N = mg\cos\alpha$ and we know that $F$ is defined as $F = \mu N$
        then
        \begin{align*}
            ma &= mg\sin\alpha - F\\
            ma &= mg\sin\alpha - \mu mg\cos\alpha\\
             a &= g(\sin\alpha - \mu\cos\alpha)
        \end{align*}
        but we can write the same equation for $a$ as
        $$a = g\cos\alpha (\tan\alpha - \mu)$$
        so if $\mu < \tan\alpha$ then $a > 0$ so the block should slide down\\
        and if $\mu > \tan\alpha$ then $a < 0$ so the block should come to rest
        at some point.
    \end{proof}
    \begin{proof}{\textbf{4.10}}
        The motion in this case is only happening in the y-axis then the
        equation of motion is given by 
        $$m \frac{dv}{dt} = -\frac{GmM}{r^2}$$
        where $r$ is the distance $OP$.\\
        Also we know that
        $$\frac{dv}{dt} = \frac{dr}{dt}\times\frac{dv}{dr} = v\frac{dv}{dr}$$
        then we have that
        \begin{align*}
            v\frac{dv}{dr} &= -\frac{GM}{r^2}\\
            \int vdv &= -\int \frac{GM}{r^2}dr\\
            \frac{v^2}{2} &= \frac{GM}{r} + C
        \end{align*}
        Where $C$ is a constant that can be determined using the initial
        conditions. If $v = \sqrt{2GM/a}$ when $r = a$ then we have
        that 
        $$C = \frac{GM}{a} - \frac{\sqrt{2GM/a}^2}{2} = 0$$
        so
        \begin{align*}
            v^2 &= \frac{2GM}{r}\\
            \left(\frac{dr}{dt}\right)^2 &= \frac{2GM}{r}\\
            \frac{dr}{dt} &= \sqrt{\frac{2GM}{r}}\\
            \int \sqrt{r}dr &= \int \sqrt{2GM}dt\\
            \frac{2\sqrt{r^3}}{3} &= \sqrt{2GM}t + C
        \end{align*}
        where $C$ is a constant that can be determined by knowing that $r=a$
        when $t = 0$, then
        $$C = \frac{2\sqrt{a^3}}{3}$$
        Finally we can write that
        \begin{align*}
            \frac{2\sqrt{r^3}}{3} &= \sqrt{2GM}t + \frac{2\sqrt{a^3}}{3}\\
            \sqrt{r^3} &= \frac{3}{2}\sqrt{2GM}t + \sqrt{a^3}\\
            r &= \left(\frac{3}{2}\sqrt{2GM}t + \sqrt{a^3}\right)^{2/3}
        \end{align*}
        Then $r$ tends to infinity when $t$ goes to infinity, therefore the
        particle $P$ escapes from $M$.
    \end{proof}
    \begin{proof}{\textbf{4.11}}
        The motion in this case is only happening in the y-axis then the
        equation of motion is given by 
        $$m \frac{dv}{dt} = -\frac{m\gamma}{r^3}$$
        where $r$ is the distance $OP$.\\
        Also we know that
        $$\frac{dv}{dt} = \frac{dr}{dt}\times\frac{dv}{dr} = v\frac{dv}{dr}$$
        then we have that
        \begin{align*}
            v\frac{dv}{dr} &= -\frac{\gamma}{r^3}\\
            \int vdv &= -\int \frac{\gamma}{r^3}dr\\
            \frac{v^2}{2} &= \frac{\gamma}{2r^2} + C
        \end{align*}
        Where $C$ is a constant that can be determined using the initial
        conditions. If $v = u$ when $r = a$ then we have
        that 
        $$C = \frac{u^2}{2} - \frac{\gamma}{2a^2}$$
        so
        \begin{align*}
            \frac{v^2}{2} &= \frac{\gamma}{2r^2} + \frac{u^2}{2} - \frac{\gamma}{2a^2}\\
            v^2 &= \frac{\gamma}{r^2} + u^2 - \frac{\gamma}{a^2}
        \end{align*}
        hence if $u^2 > \frac{\gamma}{a^2}$ we can define
        $V^2 = u^2 - \frac{\gamma}{a^2}$ where $V^2$ is a positive constant
        then
        $$v^2 = \frac{\gamma}{r^2} + V^2 > V^2$$
        since the velocity of $P$ is bigger than $V$ then $P$ escapes from $O$.\\
        
        For the case $u^2 = \gamma/2a^2$ we have that
        $$v^2 = \frac{\gamma}{r^2} - \frac{\gamma}{2a^2}$$
        and the maximum distance is reached when $v=0$ then
        \begin{align*}
            0 &= \frac{\gamma}{r_{max}^2} - \frac{\gamma}{2a^2}\\
            \frac{\gamma}{r_{max}^2} &= \frac{\gamma}{2a^2}\\
            r_{max}^2 &= 2a^2\\
            r_{max} &= \sqrt{2}a
        \end{align*}

        To find the time it takes to reach that altitude we need to get the
        trajectory from the equation of motion, then
        \begin{align*}
            v^2 &= \frac{\gamma}{r^2} - \frac{\gamma}{2a^2}\\
            \left(\frac{dr}{d\tau}\right)^2 &= \frac{\gamma}{r^2}- \frac{\gamma}{2a^2}\\
            \frac{dr}{d\tau} &= \sqrt{\frac{\gamma}{r^2}- \frac{\gamma}{2a^2}}\\
            \int_a^{\sqrt{2}a} \frac{dr}{\sqrt{\frac{\gamma}{r^2}- \frac{\gamma}{2a^2}}}
                &= \int_0^t d\tau\\
            \left[
                \frac{r\sqrt{\frac{\gamma}{r^2}-\frac{\gamma}{2a^2}}}{\frac{\gamma}{2a^2}}
            \right]_a^{\sqrt{2}a}
                &=  t\\
        \end{align*}
        in this case we are establishing the initial and final conditions
        directly into the integral 
        \begin{align*} 
            \left[
                - \frac{\sqrt{2}a\sqrt{\frac{\gamma}{2a^2}-\frac{\gamma}{2a^2}}}{\frac{\gamma}{2a^2}}
                + \frac{a\sqrt{\frac{\gamma}{a^2}-\frac{\gamma}{2a^2}}}{\frac{\gamma}{2a^2}}  
            \right]
                &=  t\\
            \frac{a\sqrt{\frac{\gamma}{2a^2}}}{\frac{\gamma}{2a^2}} &=  t\\
            \frac{\sqrt{\frac{\gamma}{2}}}{\frac{\gamma}{2a^2}} &=  t\\    
            a^2\sqrt{\frac{2}{\gamma}} &=  t
            \end{align*}
    \end{proof}
    \begin{proof}{\textbf{4.12}}
        The equation of motion in this case is happening in the y-axis
        direction, assuming y-axis is pointing from Sun to Earth, we have that
        $$m\frac{dv}{dt} = - \frac{GmM}{r^2}$$
        we are assuming $M$ and $m$ are the masses of the Sun and Earth
        respectively. Also we know that
        $$\frac{dv}{dt} = \frac{dr}{dt}\times\frac{dv}{dr} = v\frac{dv}{dr}$$
        then we can replace $dv/dt$
        \begin{align*}
            v\frac{dv}{dr} &= - \frac{GM}{r^2}\\
            \int vdv &= \int - \frac{GM}{r^2} dr\\
            \frac{v^2}{2} &= \frac{GM}{r} + C
        \end{align*}
        where $C$ is a constant that can be determined using the initial
        conditions. If $v = 0$ when $r = R$ (where $R$ is the distance at that
        moment from earth to the sun) then we have that
        $$C = - \frac{GM}{R}$$
        so we have
        \begin{align*}
            \frac{v^2}{2} &= \frac{GM}{r} - \frac{GM}{R}\\
            v^2 &= 2GM\left(\frac{1}{r} - \frac{1}{R}\right)
        \end{align*}
        now replacing $v$ with $dr/dt$ we have
        \begin{align*}
            \left(\frac{dr}{dt}\right)^2
                &= 2GM\left(\frac{1}{r} - \frac{1}{R}\right)\\
            \int_R^0 \frac{dr}{\sqrt{\frac{1}{r} - \frac{1}{R}}}
                &= \int_0^T\sqrt{2GM}dt
        \end{align*}
        in this case we have added the initial and final conditions directly to
        the integral ($T$ is the time taken for Earth to collide with the Sun),
        then
        \begin{align*}
            T &= \frac{1}{\sqrt{2GM}}\int_R^0 \frac{dr}{\sqrt{\frac{1}{r} - \frac{1}{R}}}\\
              &= \frac{1}{\sqrt{2GM}}\frac{\pi R^{3/2}}{2}\\
              &= \frac{\pi}{2}\sqrt{\frac{R^3}{2GM}}\\
        \end{align*}
        On the other hand we know that the orbital period of the earth is
        calculated as $T' = 2\pi \sqrt{\frac{R^3}{GM}}$ so we could wirte $T$
        in terms of $T'$ as
        $$T = \frac{T'}{4\sqrt{2}}$$
        since $T' = 365$ days, then $T = 64.52$ days.
    \end{proof}
    \begin{proof}{\textbf{4.13}}
        For the particle $Q$ to remain at rest we need to ensure that
        $$T = Mg$$
        where $T$ is the tension on the string and $Mg$ is the weight of $Q$.\\
        On the other hand, the particle $P$ has only a possible moving
        trajectory which is a circular motion around $O$ (the origin is fixed
        at the hole), so the equations of motion for $P$ in polar coordinates
        is given by
        $$m\left[-\frac{v^2}{b}\hatr + \dot{v}\hatth\right] = -T\hatr = -Mg\hatr$$
        where $m$ is the mass of $P$ and $v = b\ddot{\theta}$ is the
        circumferential velocity.\\
        Which taking components in the radial and transverse direction, gives
        \begin{align}
            -\frac{mv^2}{b} &= -Mg\\
            \dot{v} &= 0            
        \end{align}
        from the equation $(2)$ we see that the velocity should be constant and
        from $(1)$ we have that
        $$v^2 = \frac{Mgb}{m}$$
    \end{proof}
    \begin{proof}{\textbf{4.15}}
        We can write the equation of motion for $P$ in polar coordianates as
        $$m[(\ddot{r} - r\Omega^2)\hatr + (r\dot{\Omega} + 2\dot{r}\Omega)\hatth] = -N\hatth$$
        where $m$ is the mass of $P$ and $\Omega = \dot{\theta}$.\\
        Which taking components in the radial and transverse direction, gives
        \setcounter{equation}{0}% Restart equation counter
        \begin{align}
            \ddot{r} - r\Omega^2 &= 0\\
            r\dot{\Omega} + 2\dot{r}\Omega &= N
        \end{align}
        then equations we wanted.\\
        The equation $(1)$ is second-order linear differential equation, for
        which we propose a solution to the form
        $$r(t) = C_1e^{\Omega t} + C_2e^{-\Omega t}$$
        where $C_1$ and $C_2$ are constants that should be determined by the
        initial conditions. Since $r=a$ when $t=0$ we have that
        $$C_1 + C_2 = a$$
        and we also know that $P$ is at rest at the beginning, so derivating
        $r$ equation we get
        $$\frac{dr}{dt} = C_1\Omega e^{\Omega t} - C_2\Omega e^{-\Omega t}$$
        and applying the initial conditions $dr/dt = 0$ at $t = 0$ we get
        $$0 = C_1\Omega - C_2\Omega$$
        then $C_1 = C_2 = a/2$ from the previous equation. Therefore the
        equation for $r$ is given by
        $$r(t) = \frac{a}{2}(e^{\Omega t} + e^{-\Omega t})$$
        On the other hand, replacing $dr/dt$ in equation $(2)$ we have that
        $$N = ma\Omega^2(e^{\Omega t} - e^{-\Omega t})$$
    \end{proof}
\cleardoublepage
    \begin{proof}{\textbf{4.27}}
        With the linear resistance term included, the vector equation of motion
        for the projected body becomes
        $$m\frac{d\bm{v}}{dt} = - mK\bm{v} - mg \bm{k}$$
        the direction of $\bm{v}$ is given by
        $\bm{v} = u\cos\alpha\bm{i} + u\sin\alpha\bm{k}$. We could propose
        a different magnitude and direction for the gravity, writing the new
        direction for our new gravity as $\bm{w}$, we get that
        \begin{align*}
            g'\bm{w} &= uK\cos\alpha\bm{i} + uK\sin\alpha\bm{k} + g\bm{k}\\
                   &= uK\cos\alpha\bm{i} + (uK\sin\alpha + g)\bm{k}
        \end{align*}
        then we have that the equation of motion is given by
        $$m\frac{d\bm{v}}{dt} = -mg'\bm{w}$$
        which is equivalent to the original equation of motion but using our
        new gravity.

        On the other hand, we want to deduce that it's possible for the body to
        return to its starting point. By continuity, it's possible to end
        before and after the starting point, therefore, it's possible to end up
        exactly at the starting point, the shape of the path, in that case,
        would be an inclined line. 
    \end{proof}
    \begin{proof}{\textbf{4.29}}
        The vector equation of motion in this case is given by
        \begin{align*}
            m[(\ddot{r} - r\dot{\theta}^2)\hatr + (r\ddot{\theta} + 2\dot{r}\dot{\theta})\hatth]
                &= -T\sin{\alpha}\hatr + (T\cos{\alpha} - mg)\hatz\\
            -ma\sin\alpha\dot{\theta}^2\hatr &= -T\sin{\alpha}\hatr + (T\cos{\alpha} - mg)\hatz
        \end{align*}
        where $m$ is the mass of the particle and $T$ is the tension of the
        string.\\
        Which taking components in $\hatr$ and $\hatz$ direction, gives
        \setcounter{equation}{0}% Restart equation counter
        \begin{align}
            ma\sin\alpha\dot{\theta}^2 &= T\sin\alpha\\
            0 &= T\cos\alpha - mg
        \end{align}
        from the equation $(2)$ we have that $T = mg/\cos\alpha$ and replacing
        on equation $(1)$ we get that
        \begin{align*}
            ma\sin\alpha\dot{\theta}^2 &= mg\frac{\sin\alpha}{\cos\alpha}\\
            (a\sin\alpha)^2\dot{\theta}^2 &= ga\sin\alpha\tan\alpha\\
            u^2 &= ga\sin\alpha\tan\alpha
        \end{align*}        
    \end{proof}

\end{document}






















