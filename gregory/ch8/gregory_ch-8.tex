\documentclass[11pt]{article}
\usepackage{amssymb}
\usepackage{amsthm}
\usepackage{enumitem}
\usepackage{physics,amsmath}
\usepackage{bm}
\usepackage{adjustbox}
\usepackage{mathrsfs}
\usepackage{graphicx}
\usepackage{siunitx}
\usepackage[mathscr]{euscript}

\title{\textbf{Solved selected problems of Classical Mechanics - Gregory}}
\author{Franco Zacco}
\date{}

\addtolength{\topmargin}{-3cm}
\addtolength{\textheight}{3cm}

\newcommand{\hatr}{\bm{\hat{r}}}
\newcommand{\hatx}{\bm{\hat{x}}}
\newcommand{\haty}{\bm{\hat{y}}}
\newcommand{\hatz}{\bm{\hat{z}}}
\newcommand{\hatth}{\bm{\hat{\theta}}}
\newcommand{\hatphi}{\bm{\hat{\phi}}}
\newcommand{\hatrho}{\bm{\hat{\rho}}}
\theoremstyle{definition}
\newtheorem*{solution*}{Solution}
\renewcommand*{\proofname}{Solution}

\begin{document}
\maketitle
\thispagestyle{empty}

\section*{Chapter 8 - Non-linear oscillations and phase space}

	\begin{proof}{\textbf{8.2}}
        Let us define $s = \omega(\epsilon) t$ and change variables so
        the equation is written in terms of $s$ then
        \begin{align*}
            (\omega(\epsilon))^2 x'' + x + \epsilon x^5 = 0
        \end{align*}
        By Lindstedt's method, we know there is a solution to this equation
        in the form of the perturbation series
        $$x(s,\epsilon) = x_0(s) + \epsilon x_1(s) + \epsilon^2 x_2(s) + ...$$
        and the same for $\omega(\epsilon)$ which is also part of the solution
        $$\omega(\epsilon) = 1 + \epsilon\omega_1 + \epsilon^2\omega_2 + ...$$
        So by replacing these we get that
        \begin{align*}
            (1 + \epsilon\omega_1 + &\epsilon^2\omega_2 + ...)^2
            (x_0'' + \epsilon x_1'' + \epsilon^2 x_2'' + ...) +\\
            &(x_0 + \epsilon x_1 + \epsilon^2 x_2 + ...) +
            \epsilon(x_0 + \epsilon x_1 + \epsilon^2 x_2 + ...)^5 = 0
        \end{align*}
        Now we can equate coefficients of powers of $\epsilon$ so
        we get a succession of ODEs and initial contitions, the first two of
        which are as follows
        $$x_0'' + x_0 = 0\quad\quad (\text{zero-order})$$
        With $x_0 = 1$ and $x_0' = 0$ when $s = 0$ also we have
        $$x_1'' + x_1 = -2\omega x_0'' - x_0^5 \quad\quad (\text{first-order})$$
        With $x_1 = 0$ and $x_1' = 0$ when $s = 0$. The solution to the
        zero-order equation is
        $$x_0(s) = \cos(s)$$
        And we can substitute this in the first-order equation as follows
        \begin{align*}
            x_1'' + x_1 &= 2\omega_1 \cos(s) - \cos^5(s)\\
            x_1'' + x_1 &= \frac{1}{8}(16\omega_1 - 5)\cos(s) - \frac{5}{16}\cos(3s) - \frac{1}{16}\cos(5s)
        \end{align*}
        Where we used the identity $16\cos^5(s) = 10\cos(s) + 5\cos(3s) + \cos(5s)$.
        Also, we see that the first term of the right-hand side must be 0
        otherwise $x_1(s)$ wouldn't be periodic then
        $$\omega_1 = \frac{5}{16}$$
        This equation can now be solved by standard methods and we get that
        \begin{align*}
            x_1(s) = A\sin(s) + B\cos(s) + \frac{5}{128}\cos(3s) + \frac{1}{384}\cos(5s)
        \end{align*}
        Now by replacing with the initial conditions we get that the constant
        $B$ is
        \begin{align*}
            0 &=  B + \frac{5}{128} + \frac{1}{384}\\
            B &= -\frac{1}{24}
        \end{align*}
        And computing $x_1'(s)$ we can get the constant $A$ as
        \begin{align*}
            x_1'(s) &= A\cos(s) - B\sin(s) -\frac{15}{128}\sin(3s) -\frac{5}{384}\sin(5s) \\
            0 &= A\cos(0) - B\sin(0) -\frac{15}{128}\sin(0) -\frac{5}{384}\sin(0) \\
            A &= 0
        \end{align*}
        The solution to the first-order equation is then
        \begin{align*}
            x_1(s) = -\frac{1}{24}\cos(s) + \frac{5}{128}\cos(3s) + \frac{1}{384}\cos(5s)
        \end{align*}
        Finally, we have that when $\epsilon$ is small then
        $$\omega(\epsilon) = 1 + \frac{5}{16}\epsilon + O(\epsilon^2)$$
        and
        \begin{align*}
            x(s) = \cos(s) + \epsilon\left(-\frac{1}{24}\cos(s) +
            \frac{5}{128}\cos(3s) + \frac{1}{384}\cos(5s)\right) + O(\epsilon^2)
        \end{align*}
        where $s = (1 + \frac{5}{16}\epsilon + O(\epsilon^2))t$.
    \end{proof}
\cleardoublepage
	\begin{proof}{\textbf{8.5}}
        Let
        \begin{align*}
            \dot{x_1} = F_1(x_1,x_2,t) \quad\quad \dot{x_2} = F_2(x_1,x_2,t)
        \end{align*}
        Also, we know that $r = \sqrt{x_1^2 + x_2^2}$ then
        \begin{align*}
            \dot{r} &= \frac{x_1\dot{x_1} + x_2\dot{x_2}}{\sqrt{x_1^2 + x_2^2}}\\
            \dot{r} &= \frac{x_1F_1 + x_2F_2}{r}
        \end{align*}
        On the other hand, we know that $\tan\theta = \sin\theta/\cos\theta$ then
        \begin{align*}
            \tan\theta &= \frac{r\sin\theta}{r\cos\theta}\\
            \theta &= \tan^{-1}\frac{x_2}{x_1}
        \end{align*}
        And the derivative of $\theta$ is
        \begin{align*}
            \dot{\theta} &= \frac{x_1\dot{x_2} - x_2\dot{x_1}}{x_1^2 + x_2^2}\\
            \dot{\theta} &= \frac{x_1F_2 - x_2F_1}{r^2}
        \end{align*}

        Now let us convert the following system to the polar form
        \begin{align*}
            \dot{x} &= -x+y\\
            \dot{y} &= -x-y
        \end{align*}
        Then by using the equations we just derived we have that
        \begin{align*}
            \dot{r} &= \frac{x(-x+y) + y(-x-y)}{r}\\
            \dot{r} &= \frac{-(x^2+y^2)}{r} = \frac{-r^2}{r} = -r
        \end{align*}
        and
        \begin{align*}
            \dot{\theta} &= \frac{x(-x-y) - y(-x+y)}{r^2}\\
            \dot{\theta} &= \frac{-x^2-y^2}{r} = \frac{-r^2}{r^2} = -1
        \end{align*}
        So we have two ODEs that we can solve with standard methods to obtain
        $$r = Ae^{-t} \quad\quad \theta = -t + B$$
        As $t$ goes to infinity we see  that $\theta$ tends to $-\infty$ and $r$
        tends to $0$ so the phase path must encircle the origin and end up there
        no matter which are the initial conditions. Also, the phase paths
        rotate clockwise because the negative sign we see for $\dot{\theta}$.
    \end{proof}
\cleardoublepage
    \begin{proof}{\textbf{8.7}}
        From the damped oscillator equation
        $$\ddot{x} + \dot{x} + x = 0$$
        we can get two first-order ODEs by replacing variables as follows
        \begin{align*}
            \dot{x} &= y\\
            \dot{y} &= -y - x
        \end{align*}
        Now we can derive the polar equations by replacing values in the
        equations we determined in problem 5. We also use that $x = r\cos\theta$
        and $y= r\sin\theta$ then
        \begin{align*}
            \dot{r} &= \frac{xy + y(-y-x)}{r} = \frac{-y^2}{r}\\
            \dot{r} &= -\frac{r^2\sin^2\theta}{r} = -r\sin^2\theta
        \end{align*}
        On the other hand for $\theta$ we get that
        \begin{align*}
            \dot{\theta} &= \frac{x(-y-x) - y^2}{r^2} = -\frac{x^2 +xy+ y^2}{r^2}\\
            \dot{\theta} &= -\frac{r^2+xy}{r^2} = -(1 + \cos\theta\sin\theta)\\
            \dot{\theta} &= -\left(1 + \frac{1}{2}\sin2\theta\right)
        \end{align*} 
        Now we want to check that the phase paths encircle the origin
        infinitely many times clockwise, i.e. we want to show that
        $\theta \to -\infty$ and $r \to 0$ when $t \to \infty$. From the
        equation of $\dot{\theta}$ we see that $\dot{\theta} \leq -1/2$ then
        this implies that
        \begin{align*}
            \theta &\leq -1/2t + A
        \end{align*}
        So when $t \to \infty$ we have that $\theta \to -\infty$ no matter the
        initial conditions i.e. the phase paths encircle the origin clockwise.
        
        Finally, let us check that $r \to 0$ when $t \to \infty$ by solving the
        equation. So we see that
        \begin{align*}
            \int \frac{dr}{r} &= - \int \sin^2\theta~dt\\
            -\log(r) &= \int \sin^2\theta~dt\\
            -\log(r) &= \int \frac{\sin^2\theta}{\dot\theta}~d\theta
        \end{align*}
        Since $\dot\theta \leq -1/2$ then we can do the following analysis
        \begin{align*}
            -\log(r) &= \int \frac{\sin^2\theta}{\dot\theta}~d\theta\\
                &\geq -2\int \sin^2\theta~d\theta\\
                &= \frac{1}{2}\sin(2\theta) - \theta + C
        \end{align*}
        Since when $t \to \infty$ then $\theta \to -\infty$ we see that 
        $\frac{1}{2}\sin(2\theta) - \theta + C \to \infty$ no matter the initial
        conditions (i.e. the value of $C$). Therefore $-\log(r) \to \infty$ when
        $t \to \infty$ then $r \to 0$.
    \end{proof}
\cleardoublepage
    \begin{proof}{\textbf{8.10}}
        Consider the symmetrical predator-prey equations
        \begin{align*}
            \dot{x} &= x - xy\\
            \dot{y} &= xy - y
        \end{align*}
        We want to determine the phase path equation so we write the path
        gradient as follows
        \begin{align*}
            \frac{dx}{dy} &= \frac{x - xy}{xy - y}\\
                &= \frac{x(1-y)}{y(x-1)}
        \end{align*}
        This is a separable ODE so we can solve it
        \begin{align*}
            \int \frac{x-1}{x}~dx &= \int \frac{1-y}{y}~dy\\
            x - \log(x) &= \log(y) - y + B\\
            e^{x - \log(x)} &= e^{\log(y) - y + B}\\
            x^{-1}e^x &= e^Bye^{-y}\\
            A &= (ye^{-y})(xe^{-x})
        \end{align*}
        Where we named $A = 1/e^B$.
        Let us now consider the equation $z = (ye^{-y})(xe^{-x})$ and let us
        plot it using a computer program. We get the following plot
        \begin{center}
            \includegraphics[scale=0.3]{ch8-10.jpg}
        \end{center}
        Where we see that if we intersect the plot with a horizontal plane,
        we get a closed curve that encircles the equilibrium point (1,1) as
        we wanted. 
    \end{proof}
\cleardoublepage
    \begin{proof}{\textbf{8.11}}
        Let our system be
        \begin{align*}
            \dot{x} &= x - y - (x^2 + 4y^2)x\\
            \dot{y} &= x + y - (x^2 + 4y^2)y
        \end{align*}
        We want to show there is a limiting cycle lying in the annulus
        $1/2 < r < 1$.
        
        To find the stable points of the system we need to find the solutions to
        the following equations
        \begin{align*}
            0 &= x - y - (x^2 + 4y^2)x\\
            0 &= x + y - (x^2 + 4y^2)y
        \end{align*}
        By multiplying the first equation by $y$ and the second one by $x$ we
        get that
        \begin{align*}
            0 &= xy - y^2 - (x^2 + 4y^2)xy\\
            0 &= x^2 + xy - (x^2 + 4y^2)xy
        \end{align*}
        And subtracting the first equation from the second one we get that
        \begin{align*}
            x^2+y^2 = 0
        \end{align*}
        Therefore the only equilibrium point happens at the origin i.e. when
        $x=y=0$ so if a periodic solution exists it must enclose it.

        Let us express the system in polar coordinates as follows
        \begin{align*}
            \dot{r} &= \frac{x(x-y-(x^2 +4y^2)x) + y(x+y-(x^2+4y^2)y)}{r}\\
                &= \frac{x^2-xy-(x^2 +4y^2)x^2 + yx+y^2-(x^2+4y^2)y^2}{r}\\
                &= \frac{x^2(1-(x^2 +4y^2) + y^2(1-(x^2+4y^2))}{r}\\
                &= \frac{(1-(x^2 +4y^2))(x^2 + y^2)}{r}\\
                &= (1-(x^2 +4y^2))r\\
                &= r(1 - r^2(\cos^2\theta + 4\sin^2\theta))\\
                &= r(1 - r^2(1 + 3\sin^2\theta))
        \end{align*}
        \begin{align*}
            \dot{\theta} &= \frac{x(x+y-(x^2 +4y^2)y) - y(x-y-(x^2+4y^2)x)}{r^2}\\
            \dot{\theta} &= \frac{x^2+xy-(x^2 +4y^2)xy - yx+y^2+(x^2+4y^2)xy}{r^2}\\
            \dot{\theta} &= \frac{x^2 +y^2}{r^2} = 1
        \end{align*}
        So let us suppose the domain $\mathcal{D}$ is an annular domain where
        $1/2<r<1$. If $r = 1$ we have that
        \begin{align*}
            \dot{r} = - 3\sin^2\theta \leq 0
        \end{align*}
        So except when $\theta \in \{0, \pi, 2\pi, ...\}$ (i.e. points (1,0)
        and (-1, 0)) we have that $r < 0$,
        thus a phase point that starts anywhere on the outer boundary $r =1$
        enters the domain $\mathcal{D}$
        also if $r = 1/2$ we get that
        \begin{align*}
            \dot{r} &= \frac{1}{2}(1 - \frac{1}{4}(1 + 3 \sin^2\theta))\\
                &= \frac{3}{8}(1 - \sin^2\theta) \geq 0
        \end{align*}
        Then in the same way except when $\theta \in \{\frac{\pi}{2}, \frac{3\pi}{4}, ...\}$
        (i.e. points $(0, \pi/2)$ and $(0, -\pi/2)$) we have that $r > 0$ and so
        enters the domain $\mathcal{D}$.
        a phase point that starts anywhere on the inner boundary $r = 1/2$ also

        Since $\mathcal{D}$ is a bounded domain with no equilibrium points
        within it or on its boundaries, it follows from Poincaré-Bendixson that
        any such path must either be a simple closed loop or tend to a limit
        cycle. In any case, the solution is a periodic solution enclosed in the
        annulus. Furthermore, if the path starts at the boundaries they enter
        $\mathcal{D}$ and cannot leave it and neither close themselves otherwise
        they would be leaving $\mathcal{D}$ so they can only have a limit cycle
        inside $\mathcal{D}$.
    \end{proof}
\cleardoublepage
    \begin{proof}{\textbf{8.13}}
        Let the following driven non-linear oscillator
        $$\ddot{x} + \epsilon\dot{x}^3 + x = \cos{pt}$$
        Using perturbation theory we want to find a two terms approximation
        solution. Let us now expand $x(t,\epsilon)$ in the perturbation series
        \begin{align*}
            x(t,\epsilon) = x_0(t) + \epsilon x_1(t) + \epsilon^2 x_2(t) + ...
        \end{align*}
        And let us seek a solution to the driven non-linear oscillator that has
        period $2\pi/p$. It follows that the expansion functions
        $x_0(t), x_1(t), x_2(t), ...$ must also have period $2\pi/p$. If we now
        substitute this series into the ODE we get that
        \begin{align*}
            (\ddot{x}_0(t) + \epsilon \ddot{x}_1(t) + \epsilon^2 \ddot{x}_2(t) + ...) +& 
            \epsilon(\dot{x}_0(t) + \epsilon \dot{x}_1(t) + \epsilon^2 \dot{x}_2(t) + ...)^3 +\\
            & + (x_0(t) + \epsilon x_1(t) + \epsilon^2 x_2(t) + ...) = \cos pt
        \end{align*}
        and equating the coefficients of powers of $\epsilon$, we obtain a
        succession of ODEs. The first two of these are as follows:

        From coefficients of $\epsilon^0$ (zero-order):
        \begin{align*}
            \ddot{x}_0 + x_0 = \cos(pt) 
        \end{align*}

        From coefficients of $\epsilon^1$ (first-order):
        \begin{align*}
            \ddot{x}_1 + \dot{x}_0^3 + x_1 = 0 
        \end{align*}

        The solution to the zero-order equation is given by
        \begin{align*}
            x_0(t) = A_0\sin(t) + B_0\cos(t) - \frac{1}{p^2 -1 }\cos(pt)
        \end{align*}
        where $A_0$ and $B_0$ are arbitrary constants. Since $x_0$ is known to
        have period $2\pi/p$, it follows that $A_0$ and $B_0$ must be zero
        unless $1/p$ is an integer, we will assume that is not the case. Then
        the final solution to the zero-order equation is
        $$x_0(t) = - \frac{1}{p^2 -1 }\cos(pt)$$

        The first-order equation can now be written as
        \begin{align*}
            \ddot{x}_1 + x_1 &= -\left(\frac{p}{p^2-1}\sin(pt)\right)^3\\
            \ddot{x}_1 + x_1 &= -\frac{p^3}{4(p^2-1)^3} (3\sin(pt) - \sin(3pt))
        \end{align*}
        And the solution to this equation by taking only the terms that have a
        period of $2\pi/p$ is 
        \begin{align*}
            x_1 = \frac{3p^3\sin(pt)}{4(p^2 - 1)^4}  - \frac{p^3\sin(3 p t)}{4(p^2-1)^3(9p^2-1)}
        \end{align*}

        Finally, when $\epsilon$ is small, the driven response approximated by
        two terms is given by
        \begin{align*}
            x(t,\epsilon) = - \frac{1}{p^2 -1 }\cos(pt) +
            \epsilon\left(\frac{3p^3\sin(pt)}{4(p^2 - 1)^4} -
            \frac{p^3\sin(3 p t)}{4(p^2-1)^3(9p^2-1)}\right) + O(\epsilon^2)
        \end{align*}
        The restrictions on the value of $p$ are that $1/p$ cannot be an integer
        as we said and $p \neq 1$ and $p \neq 1/3$ otherwise this solution is
        invalid. 
    \end{proof}

\end{document}






















