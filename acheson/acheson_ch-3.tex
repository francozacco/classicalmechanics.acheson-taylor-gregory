\documentclass[11pt]{article}
\usepackage{amssymb}
\usepackage{amsthm}
\usepackage{enumitem}
\usepackage{amsmath}
\usepackage{bm}
\usepackage{adjustbox}
\usepackage{mathrsfs}

\title{\textbf{Solved selected problems of From Calculus to Chaos - Acheson}}
\author{Franco Zacco}
\date{}

\addtolength{\topmargin}{-3cm}
\addtolength{\textheight}{3cm}

\begin{document}


\maketitle
\thispagestyle{empty}

\section*{Chapter 3 - Ordinary differential equations}

	\begin{proof}{\textbf{3.2}}
		The equation presented can be classified as a nonlinear non-autonomous
		differential equation, and the equation is of the form
		$$\frac{dx}{dt}=g(x)h(t)$$
		where $g(x)=x^{2}$ and $h(t)=\frac{1}{1+t}$ which can be rewritten to
		the form
		$$\frac{1}{g(x)}\frac{dx}{dt}=h(t)$$
		as
		$$\frac{1}{x^{2}}\frac{dx}{dt} = \frac{1}{1+t}$$
		then integrating both sides with respect to $t$
		$$\int \frac{1}{x^2}\frac{dx}{dt}dt = \int \frac{1}{1+t}dt$$
		$$\int \frac{1}{x^2}dx = \int \frac{1}{1+t}dt$$
		$$-\frac{1}{x} = log(1+t) +C$$
		$$x = \frac{1}{C - log(1+t)}$$
		Subject to $x = 1$ when $t = 0$ then $C=1$ so we have
		$$x = \frac{1}{1 - log(1+t)}$$
		therefore the "blow up" of the solution happens when $t$ approaches
		$e-1$.
	\end{proof}
\cleardoublepage
	\begin{proof}{\textbf{3.4}}
		Since we know that $x = x_1u$ then
		\begin{align*}
			\dot{x} &= \dot{x_1}u + \dot{u}x_1 \\
			\ddot{x} &= \ddot{x_1}u + 2\dot{x_1}\dot{u} + x_1\ddot{u}
		\end{align*}
		by replacing in $a\ddot{x} + b\dot{x} + cx = 0$ we have that:
		\begin{align*}
			a(\ddot{x_1}u + 2\dot{x_1}\dot{u} + x_1\ddot{u}) + b(\dot{x_1}u + \dot{u}x_1) + cx_1u = 0\\
			u(a\ddot{x} + b\dot{x} + cx) + \dot{u}(2a\dot{x_1} + bx_1) + \ddot{u}ax_1 = 0
		\end{align*}
		since $a\ddot{x} + b\dot{x} + cx = 0$ then
		\begin{align*}
			\dot{u}(2a\dot{x_1} + bx_1) + \ddot{u}ax_1 = 0
		\end{align*}
		defining $z=\dot{u}$ then
		\begin{align}
			\dot{z}ax_1 + z(2a\dot{x_1} + bx_1) = 0
		\end{align}
		which is an equation of first-order for $z$.\\
		Now this method is going to be used to solve $\ddot{x}-2\dot{x}+x=0$ 
		since the solution we have is $x_1=e^{t}$ then $\dot{x_1}=e^{t}$ and
		replacing in the equation (1) with the values $a=1$ and $b=-2$
		we have that
		\begin{align*}
			\dot{z}e^{t} + z(2e^{t} -2e^{t}) = 0\\
			\dot{z}e^{t} = 0\\
			\dot{z} = 0
		\end{align*}
		then $z=B$ where $B$ is a constant, but $\dot{u}=z$ so $\dot{u}=B$ and 
		then $u = Bt + A$. Since we are assuming that $x=ux_1$ then the general
		solution is of the form
		\begin{align*}
			x &= (Bt+A)e^{t}
		\end{align*}
		then $\dot{x} = Be^{t} + (Bt+A)e^{t}$ and by applying the initial
		conditions we obtain that $A=1$ and $0 = B + A$ and then $B = -1$.\\
		Therefore the other solution is
		\begin{align*}
			x = (1-t)e^{t}
		\end{align*}
	\end{proof}
\end{document}























